\documentclass[]{article}

%opening
\title{}
\author{}

\begin{document}

\maketitle


Example: f(x) = $ 3^{-x} $ on [0,1]

\section*{Closed Interval Method}

\begin{enumerate}
	\item Critical points are x = 0 \& x = 1
	\item f'(x) = $ -3^{-x} \cdot \ln 3 \ne 0 $
	\item f' always exists
	\item $ f(0) = 1 $
	\item $ f(1) = \frac{1}{3} $
	\item $ f : [0,1] -> [\frac{1}{3}, 1] \subset [0,1] $
	\item => There is a fixed point
	\item |f'(0)| = ln3 > 1
	\item No guarantee of uniqueness
\end{enumerate}

\section*{Interpolation and Polynomial Approximation}

Interpolation refers to determining a function that exactly represents a collection of data. We need to compute values for a tabulated function at a point not in the table.

We will find a polynomial that fits a selected set of points $ (x_{i}, f(x_i)) $ \& assumes the polynomial and function are exactly the same (1st degree polynomial is called linear interpolation).

Goal is to replace some times a function by a simpler one.

\section*{Applications}

\begin{itemize}
	\item Performance of a new rocket.
	Signals are received every 10 sec. Interpolation gives the position of the rocket as well as other information. 
	\item Astronomy:
	When the motion of heavenly bodies is determined from periodic observations.
\end{itemize}

\section*{Lagrange Polynomials}

\begin{tabular}{c | c}
	 $ 	x  $ & $  f(x)   $ \\
$ 	x_0 $  &  $ f(x_0) $  \\
$ 	x_1 $  &  $ f(x_1)  $
\end{tabular}

$ P_1(x) = \frac{x-x_0}{x_0 - x_1}f_0 + \frac{x - x_1}{x_1 - x_0}f_1 $

\end{document}
