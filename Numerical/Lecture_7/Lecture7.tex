\documentclass[]{article}
\usepackage{amsmath}

%opening
\title{MTH 343 Numerical Analysis Lecture 7:}
\author{Sheikh Abdul Raheem Ali}

\begin{document}

\maketitle

\section{Method of False Position}

Bisection/Secant Method that constructs approximating lines similar to those of secant method but always brackets the root in the manner of the bisection method. 

\[ P_{n+1} = b_n - \frac{f(b_n)(b_n - a_n)}{f(b_n) - f(a_n)} \]

Example: Solve $ x^2 - 2 = 0 $ for $ x $. $ f(x) = x^2 - 2, a_1 = 1, b_1 = 2. $

\begin{tabular}{c c c c c}
	$ n $ & $ a_n $ & $ b_n $ & $ P_{n+1} $ & $ f(P_{n+1}) $ \\
	1&1&2&1.3333&-0.22222\\
	2&1.3333&2&1.4&-0.04\\
	3&1.4&2&1.4117&-0.00692\\
\end{tabular}

Although the method of False Position may appear superior to the Secant Method, it generally converges more slowly. 

\section{Newton's Method}

One of the most widely used methods.

\section*{Technique}

\begin{enumerate}
	\item Start from a single initial point (estimate) $ x_0 $ that is close to the root, then move along the tangent line to its intersection with the x-axis \& take that point as the first approximation.
	\item The procedure is continued until either the successive x-values are sufficiently close or the values of the function is sufficiently near zero.
\end{enumerate}

\begin{align*}
	\frac{f(x_0) - 0}{x_0 - x_1} &= f'(x_0) \\
	\frac{f(x_0)}{f'(x_0)} &= x_0 - x_1 \\
	x_1 &= x_0 - \frac{f(x_0)}{f'(x_0)} \\
	x_{n+1} &= x_n - \frac{f(x_n)}{f'(x_n)}
\end{align*}

This method is of order two, which means that on average it converges to a solution at roughly two additional decimal points of precision per iteration. 

It fails when it encounters a maximum or minimum (as $ f'(x_n) $ would be 0 and the tangent would be parallel to the x-axis), which means it's not as suitable for functions that oscillate. 

\section*{Remarks}

\begin{enumerate}
	\item Widely used, it converges rapidly (quadratic convergence). The number of decimal places accuracy nearly doubles at each iteration.
	\item Method may converge to a root different from the expected one. 
	\item Method may diverge if the starting value is not close enough to the root.
	\item If we reach the max or min of a curve we fly off to infinity. 
\end{enumerate}

	\section*{Theorem}
	
	Newton's method is quadratically convergent. 
	
	\section*{Proof}
	
	\[ f(x) = f(a) + f'(x)(x-x_0) + \frac{f''(x)(x-x_0)^2}{2!} + \frac{f^{(3)}(x)(x-x_0)^3}{3!} + ...  \]
	
	$ x = P $
	$ a = P_n $
	
	\begin{align*}
		f(P) = f(P_n) + f'(P_n)(P-P_n) + \frac{f''(\xi)}{2!}(P-P_n)^2 + ... \\
		0 = \frac{f(P_n)}{f'(P_n)} + P - P_n + \frac{f''(\xi)}{2f'(P_n)}(P-P_n)^2 + ...\\
		P - P_n + \frac{f(P_n)}{f'(P_n)} = - \frac{f''(\xi)}{2f'(P_n)}(P-P_n)^2 \\
		|P-P_{n+1}| = \frac{f'(\xi)}{2|f'(P_n)|}(P - P_n)^2 \\
		|P - P_{n+1}| \le \frac{M}{2f'(P_n)}|(P-P_n)^2
	\end{align*}

\end{document}
