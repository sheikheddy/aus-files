\documentclass[]{article}
\usepackage{amsmath}

\title{MTH 343 Numerical Analysis Lecture 4: Bisection Method (f(x) = 0)}
\author{Sheikh Abdul Raheem Ali}
\date{February 3, 2019}

\begin{document}
	
	\maketitle
	
	\section*{Strategy/Algorithm}
	
	Assume $ f(x) $ is continuous.
	
	\begin{enumerate}
		\item Begin with two values $ x = a \& x = b $ that bracket the root by finding $ f(a) \cdot f(b) < 0  $ (i.e they are of opposite signs).
		
		\item The method successfully divides the interval in half \& replaces one endpoint by the midpoint so that the root is again bracketed.
		
	\end{enumerate}
	
	\[ x^2 - 2 = 0, (\text{ solution } x = \sqrt{2}) \text{ on the interval} \ [1,2] \ \begin{cases}
	f(a = 1) = -1 & < 0 \\
	f(b = 2) = 2 & > 0
	\end{cases}\]
	
	\begin{tabular}{c c c c c}
		\# of iterations \textbf{$ n $} & \textbf{$ a_n $} & \textbf{$ b_n $} & midpoint \textbf{$ P_n $} & f($ P_n $) \\
		
		0&1&2&1.5&0.25 \\
		1&1&1.5&1.25& -0.4375\\
		2&1.25&1.5&1.375& -0.1094\\
		3&1.375&1.5&1.4375& +0.0664\\
		4&1.4375&1.375&1.40625& -0.0225\\
		5&1.4375&1.40625&1.421875& -0.0217\\
		6&1.40625&1.421895&1.4140625& -0.0004\\
		7&1.4140625&1.421875&1.41796875& -0.0106
		
		 
	\end{tabular}
\section*{Remarks}

\begin{enumerate}
	
	\item The main advantage of the bisection method is that it is guaranteed to work if $ f(x) $ is continuous on $ [a,b] $ and $ a,b $ bracket the root.
		
	 \item Its accuracy after n iterations is known in advance which is $ \le |\frac{b-a}{2^n}| $
	
	\[ |P_n - P| \le \frac{b-a}{2^n} \]
	
	Where $ P $ = exact value, and $ P_n $ = midpoint (approx)
	
	In the previous example, find how many iterations are needed to achieve an accuracy for $ 10^{-4}  (x^2 - 2 = 0  \text{ on }  [1,2])$.
	
	
	\begin{align*}
	 |E_n| \le \frac{2-1}{2^n} < 0.0001 \\
	 2^n > 10,000 \\
	 n \cdot \ln2 > \ln(10,000) \\
	 n > \frac{\ln(10,000)}{\ln2} \approx 13.28 \\
	 n = 14 \ terms
	\end{align*}
	
	\item A minor disadvantage is that it is slow to converge, however with speedy computers available the slowness is of less concern. 
	
	\item When multiple roots are concerned, the method may not be applicable since it might not change signs. 
\end{enumerate}	

\[ 	0 = x^2 - 6x + 9 = (x-3)^2, \  x = 3  \]
	
\end{document}