\documentclass[]{article}
\usepackage{amsmath}
\usepackage{xcolor}
\newtheorem{Def}{Definition}
%opening
\title{MTH 343 Numerical Analysis Lecture 2: Review of Computer Arithmetic}
\author{Sheikh Abdul Raheem Ali}

\begin{document}

\maketitle

\section*{Remarks}

\begin{enumerate}
	\item Numerical Analysis requires such tedious \& repetitive operations that only a computer can perform quickly \& without and mistakes.
	\item Computers are dumb and must be given complete instructions of every step. Programs can be written in any language you like. 
	\item Writing code is not very important because extensive commercial software packages are available. 
		\begin{enumerate}
			\item IMSL: International Mathematics \& Statistics Library
			\item NAG: Numerical Algorithm Group
			\item LAPACK: Linear Algebra package
		\end{enumerate}	
		Alternatives: Computer Algebra Systems
		\begin{enumerate}
			\item Mathematica
			\item Maple
			\item MATLAB
		\end{enumerate}
	
\end{enumerate}

\section*{Floating-Point Arithmetic}

In computers, numbers are stored as floating point quantities in the general form: 
\[ \pm \cdot (d_1 d_2 d_3 \ldots d_p) \cdot \beta^e, \] where
$ p $ = precision, the number of significant bits (digits), $ e $ = an integer exponent ranging from $ E_{min} $ to $ E_{max} $, $ \beta $ = the number base, normally 2, 10, 16,
$ d_i: $ ranges from 0 to $ \beta - 1 $, and $ d_1 d_2 d_3 \ldots d_p $ is called the fractional part (mantissa). 



Sometimes numbers are normalized: $ 0.023 -> 0.23 \cdot 10^{-1} $

Let us examine the case $ \beta = 10 $ (Decimal)

\begin{eqnarray*}
	3216 &=& 3\cdot10^3 + 2\cdot10^2 + 1\cdot10^1 + 6\cdot10^0\\
	&=& 10^4(3\cdot10^{-1} + 2\cdot10^{-2} + 1\cdot10^{-3} + 6\cdot10^{-4}) \\
	&=& (.3216)\cdot10^4 
\end{eqnarray*}



Now let us examine the case $ \beta = 2 $ (Binary)

\begin{eqnarray*}
65 &=& 2^6 + 2^0 \\
      &=& 2^7(2^{-1}) + 2^{-7} \\
 &=&(.1000001)_2 \cdot 2^7 
\end{eqnarray*}

\begin{eqnarray*}
	23 &=& 2^4 + 2^3 + 2^2 \\
	&=& 2^5(2^{-1} + 2^{-2} + 2^{-3}) \\
	&=& (.111)_2 \cdot 2^5
\end{eqnarray*}

\begin{eqnarray*}
	  85 &=& 2^6 + 2^4 + 2^2 + 2^0 \\
	&=& 2^7(2^{-1} + 2^{-3} + 2^{-6} + 2^{-7}) \\
	&=& (.1010011)_2 \cdot 2^7 
\end{eqnarray*}

\begin{eqnarray*}
	5.75 &=& 2^2 + 2^0 + 2^{-1} + 2^{-2} \\
	&=& 2^3(2^{-1} + 2^{-3} + 2^{-4} + 2^{5}) \\
	&=& (.10111)_2 \cdot 2^3 
\end{eqnarray*}




\[ 0.6 = (.1001100110011001\ldots)_2 \]

This last example shows us a conversion error: the decimal is recurring, but since the computer only has a finite number of bits, the value is truncated at some point.

\begin{Def}[Round off error:]
The error that is produced when a computer is used to perform real-number calculations is called round-off error.
\end{Def}

There are two ways of truncating the mantissa:

\begin{enumerate}
	\item Chopping
	\item Rounding 
\end{enumerate}

Ex. $ 13.76573 = .1376 {\bf{\color{red} |}} 573 \cdot 10^2 $

4 digits chopping: $ .1376 \cdot 10^2 $
4 digits rounding: $ .1377 \cdot 10^2 $

Numbers are \textbf{rounded} when stored in the floating point format.

\end{document}
